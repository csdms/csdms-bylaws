\documentclass[11pt, oneside]{article}   	% use "amsart" instead of "article" for AMSLaTeX format
\usepackage{geometry}                		% See geometry.pdf to learn the layout options. There are lots.
\geometry{letterpaper}                   		% ... or a4paper or a5paper or ... 
%\geometry{landscape}                		% Activate for rotated page geometry
%\usepackage[parfill]{parskip}    		% Activate to begin paragraphs with an empty line rather than an indent
\usepackage{graphicx}				% Use pdf, png, jpg, or eps§ with pdflatex; use eps in DVI mode
								% TeX will automatically convert eps --> pdf in pdflatex		
\usepackage{amssymb}

%SetFonts

%SetFonts

\def\article#1{
\renewcommand{\thesection}{Article \Roman{section}} 
\section[Article]{#1}
\renewcommand{\thesection}{\Roman{section}} 
}


\title{Community Surface Dynamics Modeling System}
\author{Bylaws}
\date{Draft, July 2024}							% Activate to display a given date or no date

\begin{document}


\maketitle



\article{Introduction}

\subsection{Purpose}

The following Bylaws will be used to govern the Community Surface Dynamics Modeling System (CSDMS). 

\subsection{About CSDMS}

The US National Science Foundation supports the CSDMS Integration Facility (CIF) to accelerate research and applications across earth-surface sciences by engaging and coordinating the scientific community, developing and curating state-of-the-art cyberinfrastructure, and providing opportunities and resources for education. CSDMS is an open, community-led organization that helps the research community collaboratively develop and use a shared, modular, and ever-improving system for computational modeling and data-model integration.


\article{Governance and Operations}

\subsection{Overview}

CSDMS is governed by the Executive Committee, which is chaired by the Executive Director, who is responsible for leading day-to-day operations, and assisted by the Deputy Director. A Steering Committee provides high-level oversight and guidance.


\subsection{CSDMS Executive Director}

The Executive Director must be an employee of the primary current award's lead academic institution and have an active, ongoing professional interest in the mission of the center. 

\subsubsection{Appointment, Term, Removal and Replacement}

The Executive Director is elected to office by a 2/3rds majority vote of the Executive and Steering Committees.  There is no term limit for the Executive Director, who may resign by filing a written notice to the CSDMS Executive Committee.  Candidates for a new Executive Director will be vetted by the CSDMS Executive Committee in consultation with the CSDMS Steering Committee.  

\subsubsection{Duties}

The Executive Director has primary responsibility for the day-to-day operations of the CIF.  The Executive Director shall: a) preside at all meetings of the CSDMS governing bodies and shall perform other duties and exercise other powers as assigned by the CSDMS Executive Committee; b) be an ex officio member of all CSDMS committees; c) is the Chief Executive Officer of the Organization, and unless authority is given by the Executive Committee to other officers or agents to do so, he or she shall execute all contracts and agreements on behalf of the Organization; d) shall be the Lead Principal Investigator on proposals that fund the core CSDMS Facility; e) see that the purposes, orders and voting within the CSDMS Organization are carried out; and, f) shall preside at any funding agency meetings and CSDMS-wide meetings.
  
\subsection{CSDMS Deputy Director}

The Deputy Director is appointed by the Executive Director and serves to advise and support the Executive Director.  In the absence of the Executive Director, the Deputy Director shall temporarily have the powers to perform the duties of the Executive Director?s office to maintain the day-to-day operations of the CSDMS Integration Facility, to preside over meetings and to perform other duties as necessary to maintain continuity for the CSDMS members and sponsors.  There is no term limit for this position.

\subsection{CSDMS Executive Committee}

The Executive Committee (ExCom) comprises a) the Executive Director and Lead PI of the award as Chair, (non-voting, except to break a tie vote); b) the Chair of the Steering Committee (voting); and, c) the Chairs of the defined Working and Focus Research Groups.
 
\subsubsection{Election, Term}

The members of ExCom (Working Group and Focus Research Group Chairs) are elected or appointed as described in Article III, Sections 3.2.3 and 3.3.3, and shall serve 3-year terms.  They are eligible for re-appointment (by election by voting group membership in the case of Working Group Chairs, and by appointment in the case of Focus Research Group Chairs). Members of ExCom other than the chair of the Steering Committee may not simultaneously serve on the Steering Committee.  

\subsubsection{Duties/Powers}
	
The ExCom is the primary decision-making body of CSDMS and will meet twice a year to provide guidance for annual planning and management, strategic planning, and other day-to-day issues that arise in the running of CSDMS. The Executive Committee will ensure that the objectives of the NSF award are met. The ExCom will develop the By-Laws and any new Operational Procedures, to be co-approved by the Steering Committee. At all meetings of ExCom, the presence of a simple majority of its Voting members then in office shall constitute a quorum for the transaction of business. So long as they do not conflict with the responsibilities of the Lead Principal Investigator (the CSDMS Executive Director), power in the management of the affairs of the CSDMS Organization is vested in the CSDMS Executive Committee. To this end, the CSDMS Executive Committee shall have the power to authorize actions on behalf of the CSDMS Organization, make rules or regulations for its management, and create additional offices or special committees. The Executive Committee shall have the power to fill vacancies in, and change the membership of committees as are constituted by it.  The CSDMS Executive Committee will co-share authority with the CSDMS Steering Committee to amend or repeal the By-Laws, or to adopt new By-Laws.

\subsubsection{Compensation}
	
No ExCom member shall be paid any compensation for serving on the CSDMS Executive Committee. Chairs may be reimbursed for the actual expenses incurred in performing duties assigned to them, within limitations of the host institution?s budget associated with the relevant funding source.

\subsection{CSDMS Steering Committee}

The CSDMS Steering Committee (SC) provides broad oversight and guidance for CSDMS. The Steering Committee has a minimum of seven (7) members.  The serving NSF program officer or designate, and the CSDMS Executive Director or designate, will serve as ex-officio members of the SC.

\subsubsection{SC Chair}\label{sec:scchair}
	
The SC Chair shall preside at all meetings of the Steering Committee and perform such other duties and exercise other powers as shall from time to time be assigned by the Executive Committee. The Chair of the Steering Committee shall be an ex officio member of all CSDMS committees.  
\subsubsection{SC Chair Election, Term}

The Chairperson of the Steering Committee shall be elected by a vote of the full CSDMS membership orchestrated and recorded by the CSDMS Program Coordinator, for a three-year term and shall be eligible for re-election.

\subsubsection{SC Member Election, Term}

Suggested nominees for open Steering Committee positions are provided by the CSDMS ExCom and should represent the spectrum of relevant Earth science and computational disciplines.  The Executive Director and the Steering Committee Chair will work together to further screen the nominees and make the final selections for open positions. The members of the Steering Committee are appointed to serve 3-year terms.  They are eligible for re-appointment at the discretion of the Executive Director and Steering Committee Chair. Members of the Steering Committee (other than the chair of the Steering Committee) may not simultaneously serve on the ExCom. 

\subsubsection{Duties/Powers}
	
The Steering Committee will meet once a year to assess the competing objectives and needs of CSDMS; will comment/advise on the progress of CSDMS efforts, management, outreach, and education; and will comment on and advise on revisions to the current strategic plan. The CSDMS Steering Committee will co-share authority with the CSDMS Executive Committee to amend or repeal the By-Laws, or the adoption of new By-Laws. During SC meetings, there may be occasions when ex-officio members would exclude themselves from discussions. The Steering Committee Chair will provide a timely annual report to the Executive Director who is to respond within four weeks.
	
\subsubsection{Compensation}
	
No SC member shall be paid any compensation for serving on the CSDMS Steering Committee. They may be reimbursed for the actual expenses incurred in performing duties assigned to them, within limitations of the host institution's budget associated with the NSF award.
	
	
\subsection{Officer Resignation, Removal and Vacancies}
 
\subsubsection{Resignation}

Any Officer (Executive Director, Deputy Director, Steering Committee Chair and Working/Focus Research Group Chairs) may resign at any time by giving written notice to the CSDMS Executive Director or to the Chairperson of the Steering Committee. Such resignation shall take effect at the time of receipt of the notice, or later specified therein. 

\subsubsection{Vacancies}

The Executive Director may fill any vacancy in any Office for the unexpired portion of the term of such office. 

\subsubsection{Removal}

Any Officer (Executive Director, Deputy Director, Steering Committee Chair and Working/Focus Research Group Chairs) may be removed at any time with cause (see CSDMS Code of Conduct, Annex~A) by a 2/3rds majority vote of the Executive Committee.


\subsection{CSDMS Integration Facility Staff}

Operations, administration, reporting, cyberinfrastructure development/maintenance, and completion of tasks outlined in the annual work plan will be conducted by the Integration Facility staff in accordance with the requirements of the primary extramural award(s) supporting CSDMS, and the rules and policies of the University of Colorado, Boulder.


\article{Membership}


\subsection{Eligibility for Membership}

CSDMS is an inclusive Earth sciences community with a purposefully broad body of members. Membership is open to US and foreign individuals. Members shall have a major commitment to research in Earth System Science with a particular emphasis on computational earth-surface dynamics, and related fields. CSDMS encourages members from all career stages, academic, government and industrial/commercial domains. Applicants may apply to the CIF to join one or more of the CSDMS Working and Focus Research Groups. The CIF shall maintain a list of Members and their Institutions available for public viewing on the CSDMS Web Portal.


\subsection{Open Membership Meetings} Each year, a variety of open meetings are available for member participation.

\subsubsection{Annual Meetings} 

An annual meeting of the members shall take place; the specific date, time, format, and location (if applicable) of which will be designated by the Executive Director.  At the annual meeting the members shall receive a report on the activities CSDMS, be provided an opportunity for input to help determine community needs, and, if necessary, vote on pertinent group business.

\subsubsection{Special Meetings} 

Special meetings may be called by the Steering Committee Chair, the Executive Director, or a simple majority of the Executive Committee. A petition signed by five percent (5\%) of voting members may also call a special meeting.

\subsubsection{Notice of Meetings} 

Written notice of Open Membership Meetings shall be given to each voting member not less than one month prior to the meeting.


\subsection{Voting Privileges}

CSDMS members in good standing are granted the privilege of voting for a replacement Steering Committee Chair (see~\S\ref{sec:scchair}) and, when requested by the Executive Director or CSDMS Executive Committee, of voting on occasional community actions where the direct input of the membership is deemed necessary to determine direction.  Any action or voting required to be taken by the CSDMS members may be taken/made electronically without an onsite meeting of the CSDMS members.


\subsection{Resignation and Removal}

Any member may resign by filing a written notice to the CSDMS Executive Director or the Director's designee.  Such resignation shall take effect at the time of receipt of the notice, or later as specified therein. Given sufficient cause, a member can have their membership terminated by an affirmative vote of 2/3rds of the voting members of the CSDMS Executive Committee.



\article{CSDMS Groups and Initiatives}

CSDMS is founded on the principle of community engagement. To that end, CSDMS has six Working Groups and six Focus Research Groups that provide members with networking, research, and educational opportunities.  Additionally, there are a number of Initiatives that are shorter-term efforts often focused on specific/narrow research questions or tasks. 


\subsection{Responsibilities and Activities of Groups}

Groups may conduct the following activities:
\begin{itemize}
\item Group Discussion: Group discussions are used to stay current in the processes and models associated with their disciplinary toolkit and identify gaps in knowledge and areas where numerical tools need to be developed. Groups also set scientific modeling priorities for their discipline and make recommendations for resource prioritization.
\item Review Activities: Groups may ensure quality control for the algorithms and modules for their area of expertise.  
\item Group Project: Groups may address a CSDMS proof-of-concept challenge, as appropriate. 
\item Individually and collectively: Group leaders and members may stimulate proposals and input from the community. They may create and/or manage the various environmental process modules related to their discipline. They also provide community continuity to meet long-term CSDMS objectives.
\item Meetings: Groups and Initiatives will coordinate much of their activity via remote communication systems, but are encouraged to meet as resources and interests permit. 
\item Reporting: Groups and Initiatives will report annually on their progress to the CSDMS Executive Director.
\end{itemize}


\subsection{CSDMS Working Groups}

The six Working Groups (WGs) that presently support the CSDMS program include three Environmental Working Groups, two Integrative Working Groups, and one Interagency Working Group.   

\subsubsection{Environmental Working Groups}

The current Environmental Working Groups are as follows: 
\begin{itemize}
\item The Terrestrial WG deals with terrestrial processes such as weathering, hillslopes, fluvial, glacial, aeolian, and lacustrine processes. 	
\item The Coastal WG focuses on deltas, estuaries, bays and lagoons, and nearshore environments. 
\item The Marine WG focuses on shelf, carbonate, slope and deep marine processes.   
\end{itemize}

\subsubsection{Integrative Working Groups}

The current Integrative Working Groups are: 
\begin{itemize}
\item  Education and Knowledge Transfer (EKT) WG: includes marketing to gain end-users, workshops to provide training for end-users, and web-based access to educational resources.  
\item  Cyber-Infrastructure and Numerics WG: includes technical computational aspects of CSDMS, ensures that the modeling system properly functions and is accessible to users; software protocols are maintained, along with model standardization and visualization.  
\end{itemize}

\subsubsection{Interagency Working Group}

The Interagency WG fosters linkages among the main US environmental agencies that have interest in CSDMS products, standards, and approaches.

\subsubsection{Working Group Membership}

Membership in the CSDMS WGs is open to all CSDMS members in good standing.  Members may belong to multiple groups.  Members may join a WG during the initial registration process or any time after membership has been granted. Members are expected to contribute to group activities when requested by the group Chair. For the purposes of electing a Working Group Chair, each CSDMS WG member shall be entitled to one vote cast electronically.  The CSDMS Integration Facility must receive votes by the deadline specified in the ballot.  All other voting on group business will be through a majority of the Working Group members present at the time of the vote (e.g., Annual Meeting of the Working Group), or through a remote voting procedure, at the discretion of the Group Chair(s).

\subsubsection{Working Group Chairs}

Each Working Group will be led by one or two Chairs who have expertise in the working group domain. Chairs of the working groups will be full voting members of the Executive Committee. They will have the authority to call meetings of the group they are responsible for and to meet the collective long-term CSDMS objectives.  

\subsubsection{Working Group Chair Elections}

Chairs of Working Groups shall be elected by the members of the respective working groups by simple majority vote, orchestrated and recorded by the CSDMS Program Coordinator, for a three-year term, and shall be eligible for re-election. In consultation with the Steering Committee Chair and the Working Group's co-Chair (if any), the Executive Director will nominate one or two candidates to stand for election for each working group chair position to be filled. Working Group members are encouraged to suggest nominees to the Executive Director. 

\subsubsection{Working Group Meetings}

Working Groups shall conduct one group meeting (with members and chairs) each year and provide feedback to the CSDMS Executive Director and to the Executive Committee during the latter's semi-annual meetings.

\subsubsection{Establishment and/or dissolution of Working Groups}

Working Groups may be established or discontinued by a majority vote of the Executive Committee. 


\subsection{CSDMS Focus Research Groups}

The CSDMS Focus Research Groups (FRGs) were established in 2008 to cut across the Environmental Working Group structure and to serve a unique subset of the surface dynamics community, often with support of well-developed co-sponsoring organizations.  The current FRGs include:      
\begin{itemize}
\item	Hydrology FRG is cosponsored by CUAHSI, the Consortium of Universities for the Advancement of Hydrologic Science, Inc.
\item	Chesapeake FRG is co-sponsored by the Chesapeake Community Modeling Program.
\item	Critical Zone FRG is co-sponsored by NSF?s Critical Zone Observatory (CZO) Program and the International Soil Modeling Consortium (ISMC).
\item	Human Dimensions FRG is co-sponsored by the AIMES (Analysis, Integration, and Modeling of the Earth System) project of the Future Earth Programme, and by CoMSES Net (the Network for Computational Modeling for Socio-Ecological Science).
\item	Geodynamics FRG (not currently sponsored by an outside organization).
\item	Ecosystem Dynamics FRG is co-sponsored by the International Society for Ecological Modelling (ISEM). 
\end{itemize}


\subsubsection{Focus Research Group Membership}

Membership in the CSDMS FRGs is open to all CSDMS members in good standing.  Members may belong to multiple groups.  Members join a FRG during the initial registration process or any time after membership has been granted. Members are expected to contribute to group activities when requested by the group Chair. Members do not vote on FRG Chairs. All other voting will be through a majority of the FRG members present at the time of the vote (e.g., Annual Meeting of the Focus Research Group). 

\subsubsection{Focus Research Group Chairs}

Each Focus Group will be led by one or two Chairs who have expertise in the working group domain. Chairs of the Focus Research Groups will be full voting members of the Executive Committee. They will have the authority to call meetings of the group they are responsible for and to meet the collective long-term CSDMS objectives. Focus Research Group Chairs are responsible for liaising between the co-sponsoring organization and the CSDMS Executive Committee.  Chairs of the FRGs report directly to the CSDMS Executive Director and, where applicable, to the Chair or Director of the co-sponsoring organization. 

\subsubsection{Focus Research Group Chair Appointments}
Focus Research Group Chairs are appointed by mutual agreement between the sponsoring organization and the CSDMS Executive Director. FRG Chairs shall serve a term of 3 years renewable by consent of the group's sponsoring organization and the CSDMS Executive Director.

\subsubsection{Focus Research Group Meetings}

Focus Research Groups shall conduct one group meeting each year and provide feedback to the CSDMS Executive Director and to the Executive Committee during the latter's semi-annual meetings.

\subsubsection{Establishment and/or dissolution of Focus Research Groups}

Focus Research Groups may be established or discontinued by a majority vote of the Executive Committee.  


\subsection{CSDMS Initiatives}

CSDMS Initiatives were established to encourage short-term, transdisciplinary research. The current Initiatives include: Coastal Vulnerability (led by the Coastal WG), Continental Margins (led by the Marine WG), Artificial Intelligence and Machine Learning (led by the Cyber WG), Exploring Interoperability of Open Modeling Platforms, River Network Modeling.  Initiatives may be established or discontinued by agreement between the Executive Director and one or more Group Chairs.


\subsection{Special or Standing Committees}

The Executive Committee and Executive Director may create special or standing committees as may be deemed desirable, the members of which shall be appointed by the Executive Director from among the Membership.  Each such committee shall have only the powers specifically delegated to it by the Executive Committee.




\article{Amendments to the Bylaws}

\subsection{Amendments to the Bylaws}

All Bylaws of the Organization shall be subject to amendment or repeal and new Bylaws may be made by the affirmative vote of two-thirds of the Executive Committee and the Steering Committee.




\newpage
\section*{Annex~A: CSDMS Code of Conduct}

[TO BE ADDED]


\end{document}